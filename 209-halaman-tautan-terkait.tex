
\subsection*{2.9 Halaman Tautan Terkait}
\addcontentsline{toc}{subsection}{2.9 Halaman Tautan Terkait}

\noindent Untuk mengatur tautan dari Aplikasi yang ditampilkan pada Landing Page Bagorganisasi, user klik Menu Tautan Terkait dan akan ditampilkan halaman Tautan Terkait seperti berikut:

\begin{figure}[H]
    \centering
    \includegraphics[width=12.45cm, height=5.75cm]{media/menu-tautan-terkait/tautan-terkait.png}
    \caption{Tampilan Menu Tautan Terkait}
    \label{fig:bagan-alir}
\end{figure}

\noindent Pada menu Tautan Terkait, ada beberapa fitur yang user dapat gunakan seperti tambah Tautan Terkait, unduh gambar, lihat detail data, ubah data, dan hapus data. Berikut tahapan setiap fitur pada Menu Tautan Terkait:

\begin{enumerate}
    \item \textbf{Tambah Tautan Terkait}\\
    Untuk melakukan tambah Tautan Terkait dapat dilakukan dengan klik tombol “Tambah Tautan Terkait”.
    
    \begin{figure}[H]
        \centering
        \includegraphics[width=12.45cm, height=5.75cm]{media/menu-tautan-terkait/tombol-tambah-tautan-terkait.png}
        \caption{Tampilan Tombol Tambah Tautan Terkait}
        \label{fig:bagan-alir}
    \end{figure}

    Setelah klik tombol “Tambah Tautan Terkait”, maka akan dialihkan ke halaman form tambah Tautan Terkait. User akan diminta untuk mengisi nama, tautan/link, dan gambar. Jika sudah klik tombol “Simpan” dan data Tautan Terkait terbaru akan tersimpan. 

    \begin{figure}[H]
        \centering
        \includegraphics[width=12.45cm, height=7.75cm]{media/menu-tautan-terkait/form-tambah-tautan-terkait.png}
        \caption{Tampilan Form Tambah Tautan Terkait}
        \label{fig:bagan-alir}
    \end{figure}

    Tautan Terkait yang sudah disimpan akan tampil pada Landing Page seperti gambar dibawah:

    \begin{figure}[H]
        \centering
        \includegraphics[width=12.45cm, height=5.75cm]{media/menu-tautan-terkait/tautan-terkait-hasil.png}
        \caption{Tampilan Hasil Tautan Terkait}
        \label{fig:bagan-alir}
    \end{figure}

    \item \textbf{Unduh Gambar}\\
    Untuk melakukan tambah Tentang Kami dapat dilakukan dengan klik icon "Unduh" maka otomatis akan mendowload gambar yang dipilih.
    
    \begin{figure}[H]
        \centering
        \includegraphics[width=12.45cm, height=5.75cm]{media/menu-tautan-terkait/unduh-gambar.png}
        \caption{Tampilan Unduh Gambar Tentang Kami}
        \label{fig:bagan-alir}
    \end{figure}

    \item \textbf{Detail Tautan Terkait}\\
    Untuk melihat detail Tautan Terkait, klik ikon mata pada kolom tabel dan akan diarahkan ke halaman detail. 

    \begin{figure}[H]
        \centering
        \includegraphics[width=12.45cm, height=5.75cm]{media/menu-tautan-terkait/icon-mata-tautan-terkait.png}
        \caption{Tampilan Icon Mata Tautan Terkait}
        \label{fig:bagan-alir}
    \end{figure}

    \begin{figure}[H]
        \centering
        \includegraphics[width=12.45cm, height=5.75cm]{media/menu-tautan-terkait/detail-tautan-terkait.png}
        \caption{Tampilan Detail Tautan Terkait}
        \label{fig:bagan-alir}
    \end{figure}

    Pada halaman detail Tautan Terkait, user dapat melihat informasi nama, tautan, gambar, dan unduh gambar. User juga dapat mengubah data Tautan Terkait dengan klik tombol "Sunting" dan kembali ke daftar Tautan Terkait dengan klik tombol "Daftar Portal Aplikasi". 

    \item \textbf{Mengubah Tautan Terkait}\\
    Untuk mengubah data Tautan Terkait, klik ikon pensil pada kolom tabel dan akan diarahkan ke halaman untuk melakukan ubah data. 

    \begin{figure}[H]
        \centering
        \includegraphics[width=12.45cm, height=5.75cm]{media/menu-tautan-terkait/icon-pensil-tautan-terkait.png}
        \caption{Tampilan Icon Pensil Tautan Terkait}
        \label{fig:bagan-alir}
    \end{figure}

    \begin{figure}[H]
        \centering
        \includegraphics[width=12.45cm, height=7.75cm]{media/menu-tautan-terkait/edit-tautan-terkait.png}
        \caption{Tampilan Form Edit Tautan Terkait}
        \label{fig:bagan-alir}
    \end{figure}

    Data yang dapat user ubah adalah nama, tautan/link, dan gambar. Jika sudah, user klik tombol "Simpan" 
    
    \item \textbf{Hapus Tautan Terkait}\\
    Untuk menghapus data Tautan Terkait, klik ikon sampah pada kolom tabel dan akan muncul pesan untuk melakukan hapus data. 

    \begin{figure}[H]
        \centering
        \includegraphics[width=12.45cm, height=5.75cm]{media/menu-tautan-terkait/hapus-tautan-terkait.png}
        \caption{Tampilan Konfirmasi Hapus Tautan Terkait}
        \label{fig:bagan-alir}
    \end{figure}

\end{enumerate}