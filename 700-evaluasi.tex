\section*{7. Evaluasi}

\addcontentsline{toc}{section}{7. Evaluasi}

\noindent Evaluasi dilakukan sebagai tahap akhir dari siklus pengembangan perangkat lunak untuk menilai sejauh mana sistem yang dibangun telah memenuhi kebutuhan dan harapan pengguna. Tahap ini penting untuk mengidentifikasi efektivitas sistem dalam mendukung proses bisnis serta menemukan kekurangan atau area yang masih dapat ditingkatkan.

\noindent Berdasarkan hasil pengujian Seluruh fitur utama seperti tambah berita, pencarian berita, dan unggah dokumen berjalan dengan baik. 

\noindent Secara keseluruhan, sistem telah mampu memenuhi kebutuhan fungsional dan memberikan pengalaman pengguna yang baik. Kekurangan yang ditemukan bersifat minor dan tidak menghambat proses kerja secara signifikan. Evaluasi juga menunjukkan bahwa sistem memiliki potensi untuk terus dikembangkan dan ditingkatkan di versi berikutnya.



