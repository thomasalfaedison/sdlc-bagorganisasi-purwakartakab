\subsubsection*{5.2.4 Menu Dokumen}

\addcontentsline{toc}{subsubsection}{5.2.4 Menu Dokumen}

\noindent \textbf{1. Filter Dokumen}\\
    Berikut table pengujian untuk filter Dokumen.
    \renewcommand{\arraystretch}{1.3}

\begin{center}
\begin{longtable}{|c|p{5cm}|c|p{5.5cm}|}
\caption{Hasil Pengujian Fitur Filter Dokumen} \label{tab:pengujian-dashboard} \\
\hline
\textbf{No} & \textbf{Fitur/Elemen yang Diuji} & \textbf{Hasil} & \textbf{Keterangan} \\
\hline
\endfirsthead

\hline
\textbf{No} & \textbf{Fitur/Elemen yang Diuji} & \textbf{Hasil} & \textbf{Keterangan} \\
\hline
\endhead

\hline
\multicolumn{4}{r}{\textit{Bersambung ke halaman berikutnya}} \\
\endfoot

\hline
\endlastfoot

1 & Dropdown kategori tampil dengan benar & Lulus & Daftar kategori dokumen muncul dan memuat semua pilihan dari database. \\
\hline
2 & Filter dokumen berdasarkan kategori & Lulus & Sistem hanya menampilkan dokumen yang sesuai dengan kategori yang dipilih. \\
\hline
3 & Filter dokumen berdasarkan nama (search) & Lulus & Sistem memunculkan dokumen yang mengandung kata kunci sesuai input. \\
\hline
4 & Filter kombinasi kategori dan nama dokumen & Lulus & Sistem mampu memfilter dokumen secara simultan berdasarkan kategori dan nama dokumen. \\
\hline
5 & Reset filter (tampilkan semua) & Lulus & Setelah filter dihapus, semua dokumen ditampilkan kembali tanpa perlu refresh halaman. \\
\hline

\end{longtable}
\end{center}

\noindent \textbf{2. Tambah Dokumen}\\
    Berikut table pengujian untuk Tambah Dokumen.
    \renewcommand{\arraystretch}{1.0}
\small

\begin{longtable}{|c|p{5cm}|c|p{5.5cm}|}
\caption{Hasil Pengujian Fitur Tambah Dokumen} \label{tab:pengujian-tambah-Dokumen} \\
\hline
\textbf{No} & \textbf{Fitur/Elemen yang Diuji} & \textbf{Hasil} & \textbf{Keterangan} \\
\hline
\endfirsthead

\hline
\textbf{No} & \textbf{Fitur/Elemen yang Diuji} & \textbf{Hasil} & \textbf{Keterangan} \\
\hline
\endhead

\hline
\endfoot

\hline
\endlastfoot

1 & Memilih kategori dokumen & Lulus & Dropdown kategori menampilkan data dengan benar dan dapat dipilih. \\
\hline
2 & Mengisi nama dokumen & Lulus & Input teks nama dokumen tersimpan dan tampil di daftar dokumen. \\
\hline
3 & Upload file dokumen & Lulus & Sistem menerima file sesuai format dan menyimpannya di direktori yang benar. \\
\hline
4 & Validasi format file saat upload & Lulus & Sistem menolak file dengan format tidak didukung. \\
\hline
5 & Memilih status publikasi (publik/privat) & Lulus & Status yang dipilih tersimpan dan sesuai saat ditampilkan. \\
\hline
6 & Menyimpan dokumen ke sistem & Lulus & Data dokumen lengkap tersimpan ke database dan muncul di daftar dokumen. \\
\hline
7 & Validasi form kosong & Lulus & Sistem menampilkan pesan kesalahan jika ada input yang belum diisi. \\
\hline

\end{longtable}

\noindent \textbf{3. Detail Dokumen} \\
    Berikut table pengujian detail Dokumen
    \begin{center}
        \small
        \renewcommand{\arraystretch}{1.0}
        
        \begin{longtable}{|c|p{5cm}|c|p{5.5cm}|}
        \caption{Hasil Pengujian Fitur Detail Dokumen} \label{tab:pengujian-detail-Dokumen} \\
        \hline
        \textbf{No} & \textbf{Fitur/Elemen yang Diuji} & \textbf{Hasil} & \textbf{Keterangan} \\
        \hline
        \endfirsthead
        
        \hline
        \textbf{No} & \textbf{Fitur/Elemen yang Diuji} & \textbf{Hasil} & \textbf{Keterangan} \\
        \hline
        \endhead
        
        \hline
        \endfoot
        
        \hline
        \endlastfoot
        
        1 & Menampilkan kategori dokumen & Lulus & Kategori ditampilkan sesuai data yang tersimpan di sistem. \\
        \hline
        2 & Menampilkan nama dokumen & Lulus & Nama dokumen ditampilkan dengan benar sesuai judul asli. \\
        \hline
        3 & Tombol unduh dokumen tersedia & Lulus & File dokumen dapat diunduh dengan format yang sesuai. \\
        \hline
        4 & Status publikasi dokumen & Lulus & Status (Ya / Tidak) ditampilkan sesuai kondisi di database. \\
        \hline
        5 & Menampilkan jumlah unduhan dokumen & Lulus & Jumlah unduhan ditampilkan secara realtime dan bertambah setiap kali dokumen diunduh. \\
        \hline
        6 & Informasi tampil lengkap dan rapi di tampilan detail & Lulus & Seluruh informasi ditampilkan dalam satu halaman dengan layout yang rapi dan mudah dibaca. \\
        \hline
        
        \end{longtable}
        \end{center}   
        
        \newpage

\noindent \textbf{4. Mengubah Dokumen} \\
    Berikut table pengujian mengubah Dokumen
    \small
    \renewcommand{\arraystretch}{1.0}
    \begin{longtable}{|c|p{5cm}|c|p{5.5cm}|}
        \caption{Hasil Pengujian Fitur Ubah Dokumen} \label{tab:pengujian-ubah-Dokumen} \\
        \hline
        \textbf{No} & \textbf{Fitur/Elemen yang Diuji} & \textbf{Hasil} & \textbf{Keterangan} \\
        \hline
    \endfirsthead

    \hline
    \textbf{No} & \textbf{Fitur/Elemen yang Diuji} & \textbf{Hasil} & \textbf{Keterangan} \\
    \hline
    \endhead

    \hline \multicolumn{4}{r}{\textit{bersambung ke halaman berikutnya}} \\
    \endfoot

    \hline
    \endlastfoot

        1 & Menampilkan data dokumen pada form edit & Lulus & Sistem menampilkan data sebelumnya secara otomatis pada semua isian (kategori, nama, file, status). \\
        \hline
        2 & Mengubah kategori dokumen & Lulus & Kategori berhasil diubah dan tersimpan dengan benar di database. \\
        \hline
        3 & Mengubah nama dokumen & Lulus & Nama dokumen dapat diubah dan ditampilkan dengan benar di halaman daftar. \\
        \hline
        4 & Mengunggah file dokumen baru & Lulus & Dokumen baru berhasil diunggah dan menggantikan file lama. \\
        \hline
        5 & Mengubah status publikasi & Lulus & Status berhasil diubah (misal dari "Draft" menjadi "Publik") dan tampil sesuai filter. \\
        \hline
        6 & Validasi jika nama/kategori kosong & Lulus & Sistem menolak penyimpanan dan menampilkan pesan kesalahan saat kolom wajib dikosongkan. \\
        \hline
        7 & Validasi format dokumen tidak sesuai (.exe, .bat) & Lulus & Sistem menolak unggahan dengan tipe file yang tidak didukung. \\
        \hline
        8 & Klik tombol simpan & Lulus & Sistem menyimpan perubahan dan menampilkan notifikasi berhasil. \\
        \hline
\end{longtable}

\newpage

\noindent \textbf{5. Hapus Dokumen} \\
    Berikut pengujian hapus Dokumen
    \begin{table}[h!]
        \centering
        \renewcommand{\arraystretch}{1.3}
        \caption{Hasil Pengujian Fitur Hapus Dokumen}
        \label{tab:pengujian-hapus-Dokumen}
        \begin{longtable}{|c|p{5cm}|c|p{5.5cm}|}
        \hline
        \textbf{No} & \textbf{Fitur/Elemen yang Diuji} & \textbf{Hasil} & \textbf{Keterangan} \\
        \hline
        1 & Klik icon hapus pada daftar Dokumen & Lulus & Sistem menampilkan konfirmasi sebelum menghapus Dokumen. \\
        \hline
        2 & Konfirmasi penghapusan disetujui & Lulus & Dokumen berhasil dihapus dari database dan tidak tampil di daftar. \\
        \hline
        3 & Konfirmasi penghapusan dibatalkan & Lulus & Penghapusan dibatalkan dan Dokumen tetap muncul di daftar. \\
        \hline
        \end{longtable}
        \end{table}
