\subsubsection*{5.2.9 Menu Tautan Terkait}

\addcontentsline{toc}{subsubsection}{5.2.9 Menu Tautan Terkait}

\noindent \textbf{1. Tambah Tautan Terkait}\\
    Berikut table pengujian untuk Tambah Tautan Terkait.
    \begin{table}[H]
        \small
        \renewcommand{\arraystretch}{0.65}
        \centering
        \caption{Hasil Pengujian Fitur Tambah Tautan Terkait}
        \label{tab:pengujian-tambah-Tautan Terkait}
        \begin{tabular}{|c|p{5cm}|c|p{5.5cm}|}
            \hline
            \textbf{No} & \textbf{Fitur/Elemen yang Diuji} & \textbf{Hasil} & \textbf{Keterangan} \\
            \hline
            1 & Menampilkan form tambah informasi "Tautan Terkait" & Lulus & Form tampil dengan kolom nama, tautan/link, dan unggah gambar. \\
            \hline
            2 & Input nama profil instansi & Lulus & Nama dapat diketik dengan bebas dan disimpan dengan baik. \\
            \hline
            3 & Input tautan eksternal (URL) & Lulus & Link dapat disimpan dan diverifikasi, serta aktif saat ditampilkan. \\
            \hline
            4 & Upload gambar profil instansi & Lulus & Gambar berhasil diunggah dan ditampilkan dalam pratinjau. \\
            \hline
            5 & Validasi form kosong & Lulus & Sistem menampilkan pesan kesalahan saat ada kolom wajib yang kosong. \\
            \hline
            6 & Simpan informasi "Tautan Terkait" & Lulus & Informasi disimpan ke database dan dapat ditampilkan di halaman publik. \\
            \hline
            \end{tabular}
        \end{table}
    
        \newpage

\noindent \textbf{2. Detail Tautan Terkait} \\
    Berikut table pengujian detail Tautan Terkait
    \begin{table}[H]
        \small
        \renewcommand{\arraystretch}{0.65}
        \centering
        \caption{Hasil Pengujian Fitur Detail Tautan Terkait}
        \label{tab:pengujian-detail-Tautan Terkait}
        \begin{tabular}{|c|p{5cm}|c|p{5.5cm}|}
            \hline
            \textbf{No} & \textbf{Fitur/Elemen yang Diuji} & \textbf{Hasil} & \textbf{Keterangan} \\
            \hline
            1 & Menampilkan nama tautan & Lulus & Nama tautan tampil sesuai data yang dimasukkan oleh admin. \\
            \hline
            2 & Menampilkan link/tautan & Lulus & Link dapat diklik dan mengarah ke halaman eksternal dengan benar. \\
            \hline
            3 & Menampilkan gambar (thumbnail/logo) & Lulus & Gambar ditampilkan sesuai data yang diunggah dan tampil proporsional. \\
            \hline
            4 & Mengunduh gambar tautan & Lulus & Gambar dapat diunduh dalam format yang sesuai (JPG/PNG) tanpa rusak. \\
            \hline
            \end{tabular}
        \end{table}

\noindent \textbf{3. Mengubah Tautan Terkait} \\
    Berikut table pengujian mengubah Tautan Terkait
    \small
\renewcommand{\arraystretch}{1.2}

\begin{longtable}{|c|p{5cm}|c|p{5.5cm}|}
\caption{Hasil Pengujian Fitur Ubah Tautan Terkait}
\label{tab:pengujian-ubah-Tautan-Terkait} \\
\hline
\textbf{No} & \textbf{Fitur/Elemen yang Diuji} & \textbf{Hasil} & \textbf{Keterangan} \\
\hline
\endfirsthead

\hline
\textbf{No} & \textbf{Fitur/Elemen yang Diuji} & \textbf{Hasil} & \textbf{Keterangan} \\
\hline
\endhead

\hline
\endfoot

\hline
\endlastfoot

1 & Menampilkan form “Tautan Terkait” & Lulus & Form menampilkan kolom nama, tautan, dan gambar dengan data yang sudah tersimpan. \\
\hline
2 & Mengubah nama & Lulus & Sistem berhasil menyimpan dan menampilkan nama yang diperbarui. \\
\hline
3 & Menambahkan/ubah tautan eksternal & Lulus & Tautan dapat diperbarui dan saat diklik mengarah ke alamat yang benar. \\
\hline
4 & Mengunggah gambar baru & Lulus & Gambar baru berhasil diunggah dan ditampilkan di halaman publik. \\
\hline
5 & Validasi form kosong & Lulus & Sistem menolak simpan jika deskripsi kosong atau link tidak valid. \\
\hline
6 & Validasi format gambar (jpg/png) & Lulus & Sistem hanya menerima format gambar tertentu dan menolak file selain gambar. \\
\hline

\end{longtable}

\vspace{-0.8em}

\noindent \textbf{4. Hapus Tautan Terkait} \\
    Berikut pengujian hapus Tautan Terkait
    \begin{table}[h!]
        \centering
        \small
        \renewcommand{\arraystretch}{1.3}
        \caption{Hasil Pengujian Fitur Hapus Tautan Terkait}
        \label{tab:pengujian-hapus-Tautan Terkait}
        \begin{tabular}{|c|p{5cm}|c|p{5.5cm}|}
        \hline
        \textbf{No} & \textbf{Fitur/Elemen yang Diuji} & \textbf{Hasil} & \textbf{Keterangan} \\
        \hline
        1 & Klik icon hapus pada daftar Tautan Terkait & Lulus & Sistem menampilkan konfirmasi sebelum menghapus Tautan Terkait. \\
        \hline
        2 & Konfirmasi penghapusan disetujui & Lulus & Tautan Terkait berhasil dihapus dari database dan tidak tampil di daftar. \\
        \hline
        3 & Konfirmasi penghapusan dibatalkan & Lulus & Penghapusan dibatalkan dan Tautan Terkait tetap muncul di daftar. \\
        \hline
        \end{tabular}
        \end{table}
