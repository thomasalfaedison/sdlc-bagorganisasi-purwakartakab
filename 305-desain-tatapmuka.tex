\subsection*{3.5 Rancangan Desain Tatap Muka}

\addcontentsline{toc}{subsection}{3.5 Rancangan Desain Tatap Muka}

Desain tatap muka dari aplikasi harus dibuat sedemikian rupa sehingga mudah untuk digunakan dan dapat memberikan informasi yang menyeluruh serta mudah untuk dibaca. 

Berikut ini beberapa rancangan desain tatap muka dari Aplikasi Bagorganisasi di Lingkungan Pemerintah Kabupaten Purwakarta.

\subsubsection*{3.5.1 Rancangan Halaman Landing Page}

\addcontentsline{toc}{subsubsection}{3.5.1 Rancangan Halaman Landing Page}

Berikut adalah Desain antarmuka untuk halaman Landing Page.

\begin{figure}[H]
    \centering
    \includegraphics[width=12.45cm, height=15.75cm]{media/bagorganisasi.drawio.png}
    \caption{Desain Tampilan Halaman Landing Page}
    \label{fig:bagan-alir}
\end{figure}

Pada halaman Landing Page ini, ditampilkan beberapa informasi seperti berita, pencapaian, dokumen, portal aplikasi, tentang kami, kontak, dan maps.

\subsubsection*{3.5.2 Rancangan Halaman Kelola Aplikasi}

\addcontentsline{toc}{subsubsection}{3.5.2 Rancangan Halaman Kelola Aplikasi}

Berikut adalah Desain antarmuka untuk halaman Kelola Aplikasi.

\begin{figure}[H]
    \centering
    \includegraphics[width=12.45cm, height=13.75cm]{media/bagorganisasi-halaman-admin.drawio.png}
    \caption{Desain Tampilan Halaman Kelola Aplikasi}
    \label{fig:bagan-alir}
\end{figure}

Pada halaman Dashboard, ditampilkan beberapa informasi seperti jumlah berita, jumlah dokumen, dan grafik aktivitas pengguna per bulan.