
\subsection*{2.6 Halaman Gambar Sampul}
\addcontentsline{toc}{subsection}{2.6 Halaman Gambar Sampul}

\noindent Untuk mengatur gambar background pada Landing Page Bagorganisasi, user klik Menu Gambar Sampul dan akan ditampilkan halaman Gambar Sampul seperti berikut:

\begin{figure}[H]
    \centering
    \includegraphics[width=12.45cm, height=5.75cm]{media/menu-gambar-sampul/gambar-sampul.png}
    \caption{Tampilan Menu Gambar Sampul}
    \label{fig:bagan-alir}
\end{figure}

\noindent Pada menu Gambar Sampul, ada beberapa fitur yang user dapat gunakan seperti tambah Gambar Sampul, unduh gambar, lihat detail data, ubah data, dan hapus data. Berikut tahapan setiap fitur pada Menu Gambar Sampul:

\begin{enumerate}
    \item \textbf{Tambah Gambar Sampul}\\
    Untuk melakukan tambah Gambar Sampul dapat dilakukan dengan klik tombol “Tambah Gambar Sampul”.
    
    \begin{figure}[H]
        \centering
        \includegraphics[width=12.45cm, height=5.75cm]{media/menu-gambar-sampul/tombol-tambah-Gambar-Sampul.png}
        \caption{Tampilan Tombol Tambah Gambar Sampul}
        \label{fig:bagan-alir}
    \end{figure}

    Setelah klik tombol “Tambah Gambar Sampul”, maka akan dialihkan ke halaman form tambah Gambar Sampul. User akan diminta untuk mengisi upload gambar dan memilih status aktif. 
    
    Status Aktif berfungsi untuk menentukan gambar mana saja yang perlu tampil pada halaman Landing Page Bagorganisasi. 
    
    Jika sudah klik tombol “Simpan” dan data Gambar Sampul terbaru akan tersimpan. 

    \begin{figure}[H]
        \centering
        \includegraphics[width=12.45cm, height=7.75cm]{media/menu-gambar-sampul/form-tambah-Gambar-Sampul.png}
        \caption{Tampilan Form Tambah Gambar Sampul}
        \label{fig:bagan-alir}
    \end{figure}

    Gambar Sampul yang sudah disimpan akan tampil pada Landing Page seperti gambar dibawah:

    \begin{figure}[H]
        \centering
        \includegraphics[width=12.45cm, height=5.75cm]{media/gambar-sampul-simpan.png}
        \caption{Tampilan Hasil Gambar Sampul}
        \label{fig:bagan-alir}
    \end{figure}

    \item \textbf{Unduh Gambar}\\
    Untuk melakukan tambah Gambar Sampul dapat dilakukan dengan klik icon "Unduh" maka otomatis akan mendowload gambar yang dipilih.
    
    \begin{figure}[H]
        \centering
        \includegraphics[width=12.45cm, height=5.75cm]{media/menu-gambar-sampul/unduh-gambar.png}
        \caption{Tampilan Unduh Gambar Gambar Sampul}
        \label{fig:bagan-alir}
    \end{figure}

    \item \textbf{Detail Gambar Sampul}\\
    Untuk melihat detail Gambar Sampul, klik ikon mata pada kolom tabel dan akan diarahkan ke halaman detail. 

    \begin{figure}[H]
        \centering
        \includegraphics[width=12.45cm, height=5.75cm]{media/menu-gambar-sampul/icon-mata-Gambar-Sampul.png}
        \caption{Tampilan Icon Mata Gambar Sampul}
        \label{fig:bagan-alir}
    \end{figure}

    \begin{figure}[H]
        \centering
        \includegraphics[width=12.45cm, height=5.75cm]{media/menu-gambar-sampul/detail-Gambar-Sampul.png}
        \caption{Tampilan Detail Gambar Sampul}
        \label{fig:bagan-alir}
    \end{figure}

    Pada halaman detail Gambar Sampul, user dapat melihat informasi Gambar Sampul dan unduh gambar. User juga dapat mengubah data Gambar Sampul dengan klik tombol "Sunting" dan kembali ke daftar Gambar Sampul dengan klik tombol "Daftar Gambar". 

    \item \textbf{Mengubah Gambar Sampul}\\
    Untuk mengubah data Gambar Sampul, klik ikon pensil pada kolom tabel dan akan diarahkan ke halaman untuk melakukan ubah data. 

    \begin{figure}[H]
        \centering
        \includegraphics[width=12.45cm, height=5.75cm]{media/menu-gambar-sampul/icon-pensil-Gambar-Sampul.png}
        \caption{Tampilan Icon Pensil Gambar Sampul}
        \label{fig:bagan-alir}
    \end{figure}

    \begin{figure}[H]
        \centering
        \includegraphics[width=12.45cm, height=7.75cm]{media/menu-gambar-sampul/edit-Gambar-Sampul.png}
        \caption{Tampilan Form Edit Gambar Sampul}
        \label{fig:bagan-alir}
    \end{figure}

    Data yang dapat user ubah adalah Gambar dan Status. Jika sudah, user klik tombol "Simpan" 
    
    \item \textbf{Hapus Gambar Sampul}\\
    Untuk menghapus data Gambar Sampul, klik ikon sampah pada kolom tabel dan akan muncul pesan untuk melakukan hapus data. 

    \begin{figure}[H]
        \centering
        \includegraphics[width=12.45cm, height=5.75cm]{media/menu-gambar-sampul/hapus-Gambar-Sampul.png}
        \caption{Tampilan Konfirmasi Hapus Gambar Sampul}
        \label{fig:bagan-alir}
    \end{figure}

\end{enumerate}