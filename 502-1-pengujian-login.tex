\subsubsection*{5.2.1 Login}

\addcontentsline{toc}{subsubsection}{5.2.1 Login}

\begin{table}[h!]
    \centering
    \renewcommand{\arraystretch}{1.3}
    \caption{Hasil Pengujian Fitur Login}
    \begin{tabular}{|c|p{5cm}|c|p{5.5cm}|}
    \hline
    \textbf{No} & \textbf{Fitur/Elemen yang Diuji} & \textbf{Hasil} & \textbf{Keterangan} \\
    \hline
    1 & Login dengan kredensial benar & Lulus & Pengguna berhasil masuk dan diarahkan ke halaman dashboard. \\
    \hline
    2 & Login dengan password salah & Lulus & Sistem menampilkan pesan kesalahan "Password salah". \\
    \hline
    3 & Login dengan Username tidak terdaftar & Lulus & Sistem menampilkan pesan "Username tidak ditemukan". \\
    \hline
    4 & Form login kosong & Lulus & Sistem menampilkan pesan validasi bahwa data wajib diisi. \\
    \hline
    5 & SQL Injection (`' OR '1'='1`) & Lulus & Input ditolak; sistem aman dari serangan SQL injection sederhana. \\
    \hline
    6 & Case sensitivity password & Lulus & Sistem membedakan huruf besar/kecil pada password. \\
    \hline
    7 & Panjang input ekstrem & Lulus & Sistem menangani input dengan panjang ekstrem tanpa error. \\
    \hline
    8 & Session login & Lulus & Session terbentuk setelah login dan hilang setelah logout. \\
    \hline
    9 & Batas percobaan login & Tidak diuji & Sistem belum dilengkapi proteksi brute-force login. \\
    \hline
    10 & Tombol logout & Lulus & Sistem berhasil logout dan menghapus session. \\
    \hline
    \end{tabular}
    \label{tab:pengujian-login}
    \end{table}