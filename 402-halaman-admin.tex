\subsection*{4.2 Halaman Kelola Aplikasi}

\addcontentsline{toc}{subsection}{4.2 Halaman Kelola Aplikasi}

\subsubsection*{4.2.1 Tampilan Dashboard}

\addcontentsline{toc}{subsubsection}{4.2.1 Tampilan Dashboard}

\noindent Berikut Tampilan Dashboard.

\begin{figure}[H]
    \centering
    \includegraphics[width=12.45cm, height=4.75cm]{media/tampilan-aplikasi/1-dashboard-admin.png}
    \caption{Tampilan Dashboard}
    \label{fig:bagan-alir}
\end{figure}

\noindent Pada gambar diatas, saat admin dengan role Admin melakukan login maka akan ditampilkan Dashboard Admin. Pada Dashboard tersebut ditampilkan informasi jumlah berita, jumlah dokumen, dan grafik statistik aktivitas pengguna perbulannya.


\subsubsection*{4.2.2 Tampilan Menu Berita}

\addcontentsline{toc}{subsubsection}{4.2.2 Tampilan Menu Berita}

\noindent Berikut Tampilan Tampilan Menu Berita.

\begin{figure}[H]
    \centering
    \includegraphics[width=12.45cm, height=4.75cm]{media/tampilan-aplikasi/2-halaman-berita.png}
    \caption{Tampilan Menu Berita}
    \label{fig:bagan-alir}
\end{figure}

\noindent Pada gambar diatas, ditampilkan halaman menu berita yang menjadi tempat admin mengelola berita seperti menambahkan berita, mengupdate berita, dan menghapus berita. Selain itu, admin juga dapat mengatur berita mana saja yang di publish atau yang tidak. Terdapat juga informasi jumlah berapa kali berita tersebut dilihat.

\subsubsection*{4.2.3 Tampilan Menu Dokumen}

\addcontentsline{toc}{subsubsection}{4.2.3 Tampilan Menu Dokumen}

\noindent Berikut Tampilan Menu Dokumen.

\begin{figure}[H]
    \centering
    \includegraphics[width=12.45cm, height=4.75cm]{media/tampilan-aplikasi/3-halaman-dokumen.png}
    \caption{Tampilan Menu Dokumen}
    \label{fig:bagan-alir}
\end{figure}

\noindent Pada gambar diatas, ditampilkan halaman menu dokumen yang menjadi tempat admin mengelola dokumen seperti menambahkan dokumen, mengupdate dokumen, dan menghapus dokumen. Selain itu, admin juga dapat mengatur dokumen mana saja yang termasuk publik atau privat. Terdapat juga informasi jumlah berapa kali dokumen tersebut diunduh.

\subsubsection*{4.2.4 Tampilan Menu Pencapaian}

\addcontentsline{toc}{subsubsection}{4.2.4 Tampilan Menu Pencapaian}

\noindent Berikut Tampilan Menu Pencapaian.

\begin{figure}[H]
    \centering
    \includegraphics[width=12.45cm, height=4.75cm]{media/tampilan-aplikasi/7-halaman-pencapaian.png}
    \caption{Tampilan Menu Pencapaian}
    \label{fig:bagan-alir}
\end{figure}

\noindent Pada gambar diatas, ditampilkan halaman menu pencapaian yang menjadi tempat admin mengelola informasi pencapaian untuk ditampilkan pada halaman Landing Page.

\subsubsection*{4.2.5 Tampilan Menu Gambar Sampul}

\addcontentsline{toc}{subsubsection}{4.2.5 Tampilan Menu Gambar Sampul}

\noindent Berikut Tampilan Menu Gambar Sampul.

\begin{figure}[H]
    \centering
    \includegraphics[width=12.45cm, height=4.75cm]{media/tampilan-aplikasi/4-halaman-gambar-sampul.png}
    \caption{Tampilan Menu Gambar Sampul}
    \label{fig:bagan-alir}
\end{figure}

\noindent Pada gambar diatas, ditampilkan halaman menu gambar sampul yang menjadi tempat admin mengelola gambar atau background pada halaman Landing Page.

\subsubsection*{4.2.6 Tampilan Menu Teks Depan}

\addcontentsline{toc}{subsubsection}{4.2.6 Tampilan Menu Teks Depan}

\noindent Berikut Tampilan Menu Teks Depan.

\begin{figure}[H]
    \centering
    \includegraphics[width=12.45cm, height=4.75cm]{media/tampilan-aplikasi/5-halaman-teks-depan.png}
    \caption{Tampilan Menu Teks Depan}
    \label{fig:bagan-alir}
\end{figure}

\noindent Pada gambar diatas, ditampilkan halaman menu teks depan yang menjadi tempat admin mengelola tulisan atau Highlight tulisan pada halama Landing Page. Admin juga dapat memasukkan suatu link atau tautan.

\newpage

\subsubsection*{4.2.6 Tampilan Menu Tentang Kami}

\addcontentsline{toc}{subsubsection}{4.2.7 Tampilan Menu Tentang Kami}

\noindent Berikut Tampilan Menu Tentang Kami.

\begin{figure}[H]
    \centering
    \includegraphics[width=12.45cm, height=4.75cm]{media/tampilan-aplikasi/8-halaman-tentang-kami.png}
    \caption{Tampilan Tentang Kami}
    \label{fig:bagan-alir}
\end{figure}

\noindent Pada gambar diatas, ditampilkan halaman menu tentang kami yang menjadi tempat admin mengelola informasi tentang kami seperti deskripsi, link tautan dan gambar yang akan ditampilkan pada halaman Landing Page.

\subsubsection*{4.2.8 Tampilan Menu Tautan Terkait}

\addcontentsline{toc}{subsubsection}{4.2.8 Tampilan Menu Tautan Terkait}

\noindent Berikut Tampilan Menu Tautan Terkait.

\begin{figure}[H]
    \centering
    \includegraphics[width=12.45cm, height=4.75cm]{media/tampilan-aplikasi/9-halaman-tautan.png}
    \caption{Tampilan Tautan Terkait}
    \label{fig:bagan-alir}
\end{figure}

\noindent Pada gambar diatas, ditampilkan halaman menu tautan terkait yang menjadi tempat admin mengelola tautan atau link dari aplikasi yang digunakan Pemerintahan Kabupaten Purwakarta untuk ditampilkan pada halaman depan Landing Page.

\subsubsection*{4.2.9 Tampilan Menu Informasi}

\addcontentsline{toc}{subsubsection}{4.2.9 Tampilan Menu Informasi}

\noindent Berikut Tampilan Menu Informasi.

\begin{figure}[H]
    \centering
    \includegraphics[width=12.45cm, height=4.75cm]{media/tampilan-aplikasi/10-halaman-informasi.png}
    \caption{Tampilan Informasi}
    \label{fig:bagan-alir}
\end{figure}

\noindent Pada gambar diatas, ditampilkan halaman menu informasi yang menjadi tempat admin mengelola informasi seperti alamat, no telepon, email, map, dan link media sosial yang akan ditampilkan pada halaman Landing Page.

\subsubsection*{4.2.10 Tampilan Menu User Admin}

\addcontentsline{toc}{subsubsection}{4.2.10 Tampilan User Admin}

\noindent Berikut Tampilan Menu User Admin.

\begin{figure}[H]
    \centering
    \includegraphics[width=12.45cm, height=4.75cm]{media/tampilan-aplikasi/6-halaman-user-admin.png}
    \caption{Tampilan admin Admin}
    \label{fig:bagan-alir}
\end{figure}

\noindent Pada gambar diatas, ditampilkan halaman menu User Admin yang menjadi tempat admin mengelola akun admin seperti tambah, ubah, dan hapus akun.

\newpage

\subsubsection*{4.2.11 Tampilan Menu User Unit Kerja}

\addcontentsline{toc}{subsubsection}{4.2.11 Tampilan User Unit Kerja}

\noindent Berikut Tampilan Menu User Unit Kerja.

\begin{figure}[H]
    \centering
    \includegraphics[width=12.45cm, height=4.75cm]{media/tampilan-aplikasi/6-halaman-user-unit-kerja.png}
    \caption{Tampilan Menu User Unit Kerja}
    \label{fig:bagan-alir}
\end{figure}

\noindent Pada gambar diatas, ditampilkan halaman menu User Unit Kerja yang menjadi tempat admin mengelola akun User Unit Kerja seperti tambah, ubah, dan hapus akun.