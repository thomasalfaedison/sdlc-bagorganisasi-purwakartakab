\subsubsection*{5.2.5 Menu Pencapaian}

\addcontentsline{toc}{subsubsection}{5.2.5 Menu Pencapaian}

\noindent \textbf{1. Tambah Pencapaian}\\
    Berikut table pengujian untuk Tambah Pencapaian.
    \renewcommand{\arraystretch}{0.75}
\begin{longtable}{|c|p{5cm}|c|p{5.5cm}|}
\caption{Hasil Pengujian Fitur Tambah Pencapaian} \label{tab:pengujian-tambah-Pencapaian} \\
\hline
\textbf{No} & \textbf{Fitur/Elemen yang Diuji} & \textbf{Hasil} & \textbf{Keterangan} \\
\hline
\endfirsthead

\hline
\textbf{No} & \textbf{Fitur/Elemen yang Diuji} & \textbf{Hasil} & \textbf{Keterangan} \\
\hline
\endhead

\hline
\multicolumn{4}{r}{\textit{bersambung ke halaman berikutnya}} \\
\endfoot

\hline
\endlastfoot

1 & Form tambah pencapaian tampil & Lulus & Halaman form tampil dengan kolom judul dan unggah gambar. \\
\hline
2 & Input judul valid & Lulus & Sistem menerima judul dan menyimpannya dengan benar. \\
\hline
3 & Input gambar valid & Lulus & Gambar berhasil diunggah dan disimpan. \\
\hline
4 & Tombol simpan ditekan dengan isian lengkap & Lulus & Data tersimpan ke database dan muncul di daftar pencapaian. \\
\hline
5 & Form disimpan tanpa mengisi judul & Lulus & Sistem menampilkan pesan kesalahan “Judul wajib diisi”. \\
\hline
6 & Form disimpan tanpa memilih gambar & Lulus & Sistem menampilkan pesan kesalahan “Gambar wajib diunggah”. \\
\hline
7 & Upload file dengan format tidak valid (PDF, DOCX) & Lulus & Sistem menolak file dan menampilkan pesan kesalahan “Format file tidak didukung”. \\
\hline

\end{longtable}
\noindent \textbf{2. Detail Pencapaian} \\
    Berikut table pengujian detail Pencapaian
    \begin{table}[H]
        \small
        \renewcommand{\arraystretch}{0.75}
        \centering
        \caption{Hasil Pengujian Fitur Detail Pencapaian}
        \label{tab:pengujian-detail-Pencapaian}
        \begin{tabular}{|c|p{5cm}|c|p{5.5cm}|}
            \hline
            \textbf{No} & \textbf{Fitur/Elemen yang Diuji} & \textbf{Hasil} & \textbf{Keterangan} \\
            \hline
            1 & Menampilkan judul pencapaian & Lulus & Judul pencapaian tampil sesuai data yang disimpan di database. \\
            \hline
            2 & Menampilkan gambar pencapaian & Lulus & Gambar tampil dengan ukuran dan kualitas yang sesuai pada halaman detail. \\
            \hline
            3 & Tombol unduh gambar tersedia & Lulus & Terdapat tombol unduh di bawah gambar atau di pojok atas tampilan. \\
            \hline
            4 & Fungsi unduh gambar berjalan baik & Lulus & Gambar berhasil diunduh ke perangkat pengguna tanpa korupsi. \\
            \hline
            \end{tabular}
        \end{table}

\noindent \textbf{3. Mengubah Pencapaian} \\
    Berikut table pengujian mengubah Pencapaian
    \small
\renewcommand{\arraystretch}{0.75}
\begin{center}
\begin{longtable}{|c|p{5cm}|c|p{5.5cm}|}
\caption{Hasil Pengujian Fitur Ubah Pencapaian} \label{tab:pengujian-ubah-Pencapaian} \\
\hline
\textbf{No} & \textbf{Fitur/Elemen yang Diuji} & \textbf{Hasil} & \textbf{Keterangan} \\
\hline
\endfirsthead

\hline
\textbf{No} & \textbf{Fitur/Elemen yang Diuji} & \textbf{Hasil} & \textbf{Keterangan} \\
\hline
\endhead

\hline
\multicolumn{4}{r}{\textit{lanjutan ke halaman berikutnya}} \\
\endfoot

\hline
\endlastfoot

1 & Menampilkan data pencapaian yang akan diubah & Lulus & Data judul dan gambar ditampilkan dengan benar sesuai data yang dipilih. \\
\hline
2 & Mengubah judul pencapaian & Lulus & Sistem berhasil menyimpan perubahan judul. \\
\hline
3 & Mengunggah ulang gambar pencapaian & Lulus & Gambar baru berhasil diunggah dan menggantikan gambar sebelumnya. \\
\hline
4 & Validasi jika judul kosong & Lulus & Sistem menampilkan pesan error bahwa judul wajib diisi. \\
\hline
5 & Validasi file gambar & Lulus & Sistem menolak file non-gambar dan menampilkan pesan kesalahan. \\
\hline
6 & Tombol simpan perubahan & Lulus & Setelah klik simpan, data berhasil diperbarui dan ditampilkan kembali dengan data terbaru. \\
\hline

\end{longtable}
\end{center}

\noindent \textbf{4. Hapus Pencapaian} \\
    Berikut pengujian hapus Pencapaian
    \begin{table}[h!]
        \centering
        \small
        \renewcommand{\arraystretch}{1.3}
        \caption{Hasil Pengujian Fitur Hapus Pencapaian}
        \label{tab:pengujian-hapus-Pencapaian}
        \begin{tabular}{|c|p{5cm}|c|p{5.5cm}|}
        \hline
        \textbf{No} & \textbf{Fitur/Elemen yang Diuji} & \textbf{Hasil} & \textbf{Keterangan} \\
        \hline
        1 & Klik icon hapus pada daftar Pencapaian & Lulus & Sistem menampilkan konfirmasi sebelum menghapus Pencapaian. \\
        \hline
        2 & Konfirmasi penghapusan disetujui & Lulus & Pencapaian berhasil dihapus dari database dan tidak tampil di daftar. \\
        \hline
        3 & Konfirmasi penghapusan dibatalkan & Lulus & Penghapusan dibatalkan dan Pencapaian tetap muncul di daftar. \\
        \hline
        \end{tabular}
        \end{table}
