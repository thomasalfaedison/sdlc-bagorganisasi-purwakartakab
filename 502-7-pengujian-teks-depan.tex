\subsubsection*{5.2.7 Menu Teks Depan}

\addcontentsline{toc}{subsubsection}{5.2.7 Menu Teks Depan}

\noindent \textbf{1. Tambah Teks Depan}\\
    Berikut table pengujian untuk Tambah Teks Depan.
    \begin{table}[H]
        \small
        \renewcommand{\arraystretch}{0.75}
        \centering
        \caption{Hasil Pengujian Fitur Tambah Teks Depan}
        \label{tab:pengujian-tambah-Teks Depan}
        \begin{tabular}{|c|p{5cm}|c|p{5.5cm}|}
            \hline
            \textbf{No} & \textbf{Fitur/Elemen yang Diuji} & \textbf{Hasil} & \textbf{Keterangan} \\
            \hline
            1 & Form input teks utama & Lulus & Kolom teks utama berhasil diisi dan ditampilkan di halaman depan. \\
            \hline
            2 & Form input subteks & Lulus & Subteks berhasil disimpan dan ditampilkan di bawah teks utama. \\
            \hline
            3 & Form input teks tombol & Lulus & Teks tombol tampil sesuai input, seperti "Selengkapnya". \\
            \hline
            4 & Form input tautan/link & Lulus & Tombol mengarah ke tautan yang sesuai saat diklik. \\
            \hline
            5 & Pilihan status aktif/tidak aktif & Lulus & Saat status tidak aktif, konten tidak ditampilkan di halaman depan. \\
            \hline
            6 & Validasi input kosong & Lulus & Sistem menampilkan pesan validasi jika ada kolom wajib yang kosong. \\
            \hline
            7 & Simpan teks depan ke database & Lulus & Data berhasil disimpan dan dapat ditampilkan ulang atau diperbarui. \\
            \hline
            8 & Teks tampil realtime di landing page & Lulus & Perubahan teks langsung terlihat tanpa error setelah disimpan. \\
            \hline
            \end{tabular}
        \end{table}

        \newpage
    
\noindent \textbf{2. Detail Teks Depan} \\
    Berikut table pengujian detail Teks Depan
    \begin{table}[H]
        \small
        \renewcommand{\arraystretch}{0.75}
        \centering
        \caption{Hasil Pengujian Fitur Detail Teks Depan}
        \label{tab:pengujian-detail-Teks Depan}
        \begin{tabular}{|c|p{5cm}|c|p{5.5cm}|}
            \hline
            \textbf{No} & \textbf{Fitur/Elemen yang Diuji} & \textbf{Hasil} & \textbf{Keterangan} \\
            \hline
            1 & Menampilkan Teks & Lulus & Teks tampil yang sesuai pada halaman detail. \\
            \hline
            2 & Menampilkan Sub Teks & Lulus & Sub Teks tampil yang sesuai pada halaman detail. \\
            \hline
            3 & Menampilkan Teks Tombol & Lulus & Teks Tombol tampil yang sesuai pada halaman detail. \\
            \hline
            4 & Menampilkan Tautan/Link & Lulus & Tautan/Link tampil yang sesuai pada halaman detail. \\
            \hline
            \end{tabular}
        \end{table}

\noindent \textbf{3. Mengubah Teks Depan} \\
    Berikut table pengujian mengubah Teks Depan
    \smallskip % Untuk spasi kecil sebelum tabel

\renewcommand{\arraystretch}{1} % Spasi antarbaris, bisa ubah ke 0.9 kalau mau lebih rapat
\begin{center}
\begin{longtable}{|c|p{5cm}|c|p{5.5cm}|}
\caption{Hasil Pengujian Fitur Ubah Teks Depan}
\label{tab:pengujian-ubah-Teks-Depan} \\
\hline
\textbf{No} & \textbf{Fitur/Elemen yang Diuji} & \textbf{Hasil} & \textbf{Keterangan} \\
\hline
\endfirsthead

\hline
\textbf{No} & \textbf{Fitur/Elemen yang Diuji} & \textbf{Hasil} & \textbf{Keterangan} \\
\hline
\endhead

\hline
\endfoot

\hline
\endlastfoot

1 & Menampilkan form ubah teks depan & Lulus & Form ditampilkan dengan isian: teks, subteks, teks tombol, tautan, dan status. \\
\hline
2 & Input teks utama & Lulus & Sistem menyimpan teks utama dengan benar dan langsung ditampilkan di halaman depan. \\
\hline
3 & Input subteks & Lulus & Subteks berhasil disimpan dan ditampilkan tepat di bawah teks utama. \\
\hline
4 & Input teks tombol & Lulus & Tombol di halaman depan menampilkan teks sesuai input. \\
\hline
5 & Input tautan/link tombol & Lulus & Saat tombol ditekan, pengguna diarahkan ke tautan yang dimasukkan. \\
\hline
6 & Ubah status aktif menjadi tidak aktif & Lulus & Jika status dinonaktifkan, komponen teks depan tidak muncul di halaman landing. \\
\hline
7 & Validasi isian kosong & Lulus & Sistem menolak penyimpanan jika isian penting (teks/tombol/link) dikosongkan. \\
\hline
\end{longtable}
\end{center}

\vspace{-0.5em}

\noindent \textbf{4. Hapus Teks Depan} \\
    Berikut pengujian hapus Teks Depan
    \begin{table}[h!]
        \centering
        \small
        \renewcommand{\arraystretch}{1.3}
        \caption{Hasil Pengujian Fitur Hapus Teks Depan}
        \label{tab:pengujian-hapus-Teks Depan}
        \begin{tabular}{|c|p{5cm}|c|p{5.5cm}|}
        \hline
        \textbf{No} & \textbf{Fitur/Elemen yang Diuji} & \textbf{Hasil} & \textbf{Keterangan} \\
        \hline
        1 & Klik icon hapus pada daftar Teks Depan & Lulus & Sistem menampilkan konfirmasi sebelum menghapus Teks Depan. \\
        \hline
        2 & Konfirmasi penghapusan disetujui & Lulus & Teks Depan berhasil dihapus dari database dan tidak tampil di daftar. \\
        \hline
        3 & Konfirmasi penghapusan dibatalkan & Lulus & Penghapusan dibatalkan dan Teks Depan tetap muncul di daftar. \\
        \hline
        \end{tabular}
        \end{table}
