\newpage
\subsubsection*{2.11.2 Halaman Unit Kerja}
\addcontentsline{toc}{subsubsection}{2.11.2 Halaman Unit Kerja}

\noindent Untuk mengelola data User Unit Kerja, klik Menu User lalu pilih Sub Menu Unit Kerja dan akan ditampilkan halaman User Unit Kerja seperti berikut:

\begin{figure}[H]
    \centering
    \includegraphics[width=12.45cm, height=5.75cm]{media/menu-unit-kerja/unit-kerja.png}
    \caption{Tampilan Sub Menu Unit Kerja}
    \label{fig:bagan-alir}
\end{figure}

\noindent Pada Sub Menu Unit Kerja, ada beberapa fitur yang user dapat gunakan seperti filter data, tambah data, ubah password, ubah data, dan hapus data. Berikut tahapan setiap fitur pada Menu User:

\begin{enumerate}
    \item \textbf{Filter Username}\\
    Untuk filter data berdasarkan Username, klik kolom isian Username pada Card Filter, 
    lalu isikan Username yang ingin ditampilkan dan klik tombol “Filter” seperti gambar 
    dibawah. 

    \begin{figure}[H]
        \centering
        \includegraphics[width=12.45cm, height=5.75cm]{media/menu-unit-kerja/filter-user.png}
        \caption{Tampilan Filter User Unit Kerja}
        \label{fig:bagan-alir}
    \end{figure}

    Jika ingin melakukan filter berdasarkan Instansi, klik dropdown Instansi dan akan ditampilkan pilihan Instansi berita seperti gambar dibawah:

    \begin{figure}[H]
        \centering
        \includegraphics[width=12.45cm, height=5.75cm]{media/menu-unit-kerja/dropdown-instansi.png}
        \caption{Tampilan Dropdown Filter Instansi}
        \label{fig:bagan-alir}
    \end{figure}

    Pilih salah satu Instansi lalu klik filter, maka akan ditampilkan data sesuai dengan dropdown yang dipilih.

    \begin{figure}[H]
        \centering
        \includegraphics[width=12.45cm, height=5.75cm]{media/menu-unit-kerja/hasil-filter-instansi.png}
        \caption{Tampilan Hasil Filter Instansi}
        \label{fig:bagan-alir}
    \end{figure}

    \newpage
    
    \item \textbf{Tambah User}\\
    Untuk melakukan tambah User dapat dilakukan dengan klik tombol “Tambah User”.
    
    \begin{figure}[H]
        \centering
        \includegraphics[width=12.45cm, height=5.75cm]{media/menu-unit-kerja/tombol-tambah-User.png}
        \caption{Tampilan Tombol Tambah User}
        \label{fig:bagan-alir}
    \end{figure}

    Setelah klik tombol “Tambah User”, maka akan dialihkan ke halaman form tambah User. User akan diminta untuk mengisi nama, username, dan password. Jika sudah klik tombol “Simpan” dan data User terbaru akan tersimpan. 

    \begin{figure}[H]
        \centering
        \includegraphics[width=12.45cm, height=5.75cm]{media/menu-unit-kerja/form-tambah-User.png}
        \caption{Tampilan Form Tambah User}
        \label{fig:bagan-alir}
    \end{figure}

    \item \textbf{Ubah Password}\\
    Jika ada user yang lupa Password dan ingin me-reset atau mengubah Password, dapat melalui permintaan ke Unit Kerja. Untuk mengubah password akun Unit Kerja tertentu, klik ikon kunci pada kolom Set Password dan akan diarahkan ke halaman Ubah Password. 

    \begin{figure}[H]
        \centering
        \includegraphics[width=12.45cm, height=5.75cm]{media/menu-unit-kerja/icon-mata-User.png}
        \caption{Tampilan Icon Kunci User}
        \label{fig:bagan-alir}
    \end{figure}

    \begin{figure}[H]
        \centering
        \includegraphics[width=12.45cm, height=5.75cm]{media/menu-unit-kerja/detail-User.png}
        \caption{Tampilan Ubah Password}
        \label{fig:bagan-alir}
    \end{figure}

    Pada halaman ubah password, User akan diminta untuk mengisi password baru dan konfirmasi password baru. Jika sudah klik tombol “Simpan” dan password user tersebut sudah berhasil diubah.  

    \item \textbf{Mengubah User}\\
    Untuk mengubah data User, klik ikon pensil pada kolom tabel dan akan diarahkan ke halaman untuk melakukan ubah data. 

    \begin{figure}[H]
        \centering
        \includegraphics[width=12.45cm, height=5.75cm]{media/menu-unit-kerja/icon-pensil-User.png}
        \caption{Tampilan Icon Pensil User}
        \label{fig:bagan-alir}
    \end{figure}

    \begin{figure}[H]
        \centering
        \includegraphics[width=12.45cm, height=5.75cm]{media/menu-unit-kerja/edit-User.png}
        \caption{Tampilan Form Edit User}
        \label{fig:bagan-alir}
    \end{figure}

    Data yang dapat user ubah adalah nama dan username. Sebelum klik tombol "Simpan", kolom isian password harus diisi dengan password akun yang diubah, jika password salah maka gagal untuk mengubah data user yang sebelumnya sudah diisi. Jika sudah menyesuaikan data dan mengisi password yang benar, user klik tombol "Simpan" dan data akan berubah sesuai kolom isian.
    
    \item \textbf{Hapus User}\\
    Untuk menghapus data User, klik ikon sampah pada kolom tabel dan akan muncul pesan untuk melakukan hapus data. 

    \begin{figure}[H]
        \centering
        \includegraphics[width=12.45cm, height=5.75cm]{media/menu-unit-kerja/hapus-User.png}
        \caption{Tampilan Konfirmasi Hapus User}
        \label{fig:bagan-alir}
    \end{figure}

\end{enumerate}