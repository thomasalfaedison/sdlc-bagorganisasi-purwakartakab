
\subsection*{2.4 Halaman Dokumen}
\addcontentsline{toc}{subsection}{2.4 Halaman Dokumen}

\noindent Pada menu Dokumen, ada beberapa fitur yang user dapat gunakan seperti filter data, tambah data (Dokumen dan Kategori), download file, lihat detail data, ubah data, dan hapus data. Berikut tahapan setiap fitur pada Menu Dokumen:

\begin{enumerate}
    \item \textbf{Filter Dokumen}\\
    Untuk filter data berdasarkan judul Dokumen, klik kolom isian Nama pada Card Filter, 
    lalu isikan nama salah satu Dokumen yang ingin ditampilkan dan klik tombol “Filter” seperti gambar 
    dibawah. 

    \begin{figure}[H]
        \centering
        \includegraphics[width=12.45cm, height=5.75cm]{media/menu-dokumen/filter-dokumen.png}
        \caption{Tampilan Filter Dokumen}
        \label{fig:bagan-alir}
    \end{figure}

    Jika ingin melakukan filter berdasarkan kategori berita, klik dropdown kategori dan akan ditampilkan pilihan kategori berita seperti gambar dibawah:

    \begin{figure}[H]
        \centering
        \includegraphics[width=12.45cm, height=5.75cm]{media/menu-dokumen/kategori-berita.png}
        \caption{Tampilan Filter Dokumen}
        \label{fig:bagan-alir}
    \end{figure}

    Pilih salah satu kategori lalu klik filter, maka akan ditampilkan data sesuai dengan dropdown yang dipilih.

    \begin{figure}[H]
        \centering
        \includegraphics[width=12.45cm, height=5.75cm]{media/menu-dokumen/hasil-filter-kategori.png}
        \caption{Tampilan Filter Dokumen}
        \label{fig:bagan-alir}
    \end{figure}
    
    \item \textbf{Tambah Dokumen}\\
    Untuk melakukan tambah Dokumen dapat dilakukan dengan klik tombol “Tambah Dokumen”.
    
    \begin{figure}[H]
        \centering
        \includegraphics[width=12.45cm, height=5.75cm]{media/menu-dokumen/tombol-tambah-Dokumen.png}
        \caption{Tampilan Tombol Tambah Dokumen}
        \label{fig:bagan-alir}
    \end{figure}

    Setelah klik tombol “Tambah Dokumen”, maka akan dialihkan ke halaman form tambah Dokumen. User akan diminta untuk memilih kategori, Nama Dokumen, upload dokumen, dan pilih status publikasi. 
    
    Status publikasi berfungsi untuk menentukan dokumen tersebut ditampilkan pada Landing Page atau tidak ditampilkan. 
    
    Jika sudah, klik tombol “Simpan” dan data Dokumen terbaru akan tersimpan. 

    \begin{figure}[H]
        \centering
        \includegraphics[width=12.45cm, height=5.75cm]{media/menu-dokumen/form-tambah-Dokumen.png}
        \caption{Tampilan Form Tambah Dokumen}
        \label{fig:bagan-alir}
    \end{figure}

    Jika Status Publikasi "Yes" maka akan tampil pada Landing Page bagian Dokumen seperti gambar dibawah:

    \begin{figure}[H]
        \centering
        \includegraphics[width=12.45cm, height=5.75cm]{media/dokumen.png}
        \caption{Tampilan Hasil Dokumen}
        \label{fig:bagan-alir}
    \end{figure}

    \item \textbf{Kategori Dokumen}\\
    Pada menu Dokumen, user dapat mengelola data kategori dokumen mulai dari tambah data, lihat data, ubah data, dan hapus data dengan cara klik tombol "Kategori", maka akan dialihkan ke halaman Kategori. Berikut tahapan pengelolaan data Kategori Dokumen:

    \begin{enumerate}
        \item Tambah Kategori\\
        Klik tombol "Kategori" dan akan dialihkan ke halaman Kategori
        \begin{figure}[H]
            \centering
            \includegraphics[width=12.45cm, height=5.75cm]{media/menu-dokumen/tombol-kategori.png}
            \caption{Tampilan Tombol Kategori}
            \label{fig:bagan-alir}
        \end{figure}
    
        \begin{figure}[H]
            \centering
            \includegraphics[width=12.45cm, height=5.75cm]{media/menu-dokumen/halaman-kategori.png}
            \caption{Tampilan Halaman Kategori}
            \label{fig:bagan-alir}
        \end{figure}

        Klik tombol "Tambah Kategori" maka user akan dialihkan ke form untuk menambahkan data baru kategori dokumen.

        \begin{figure}[H]
            \centering
            \includegraphics[width=12.45cm, height=5.75cm]{media/menu-dokumen/tombol-tambah-kategori.png}
            \caption{Tampilan Tombol Kategori}
            \label{fig:bagan-alir}
        \end{figure}
    
        \begin{figure}[H]
            \centering
            \includegraphics[width=12.45cm, height=5.75cm]{media/menu-dokumen/form-tambah-kategori.png}
            \caption{Tampilan Form Tambah Kategori}
            \label{fig:bagan-alir}
        \end{figure}

        User mengisi kolom isian nama kategori yang ingin ditambahkan. Jika sudah, klik tombol "Simpan" maka kategori baru akan tersimpan
          
        \item Edit Kategori\\
        Untuk mengubah data berita, klik ikon pensil pada kolom tabel dan akan diarahkan ke halaman untuk melakukan ubah data. 

        \begin{figure}[H]
            \centering
            \includegraphics[width=12.45cm, height=5.75cm]{media/menu-dokumen/icon-ubah-kategori.png}
            \caption{Tampilan Icon Pensil Kategori}
            \label{fig:bagan-alir}
        \end{figure}

        \begin{figure}[H]
            \centering
            \includegraphics[width=12.45cm, height=5.75cm]{media/menu-dokumen/form-edit-kategori.png}
            \caption{Tampilan Form Edit Kategori}
            \label{fig:bagan-alir}
        \end{figure}

        User mengubah nama kategori sesuai dengan yang dibutuhkan. Jika sudah sesuai, user klik tombol "Simpan" 
        
        \item Hapus Kategori\\
        Untuk menghapus data kategori, klik ikon sampah pada kolom tabel dan akan muncul pesan untuk melakukan hapus data. 

        \begin{figure}[H]
            \centering
            \includegraphics[width=12.45cm, height=5.75cm]{media/menu-dokumen/hapus-kategori.png}
            \caption{Tampilan Konfirmasi Hapus Kategori}
            \label{fig:bagan-alir}
        \end{figure}
        
        \end{enumerate}


    \item \textbf{Download Dokumen}\\
    Untuk download Dokumen, klik ikon unduh pada kolom file dan akan otomatis mendownload file dokumen. 
    
    \begin{figure}[H]
        \centering
        \includegraphics[width=12.45cm, height=5.75cm]{media/menu-dokumen/unduh-dokumen.png}
        \caption{Tampilan Icon Unduh Dokumen}
        \label{fig:bagan-alir}
    \end{figure}

    \item \textbf{Detail Dokumen}\\
    Untuk melihat detail Dokumen, klik ikon mata pada kolom tabel dan akan diarahkan ke halaman detail. 

    \begin{figure}[H]
        \centering
        \includegraphics[width=12.45cm, height=5.75cm]{media/menu-dokumen/icon-mata-dokumen.png}
        \caption{Tampilan Icon Mata Dokumen}
        \label{fig:bagan-alir}
    \end{figure}

    \begin{figure}[H]
        \centering
        \includegraphics[width=12.45cm, height=5.75cm]{media/menu-dokumen/detail-Dokumen.png}
        \caption{Tampilan Detail Dokumen}
        \label{fig:bagan-alir}
    \end{figure}

    Pada halaman detail Dokumen, user dapat melihat informasi kategori, nama dokumen, unduh file dokumen, melihat status publik, dan jumlah unduhan dokumen. User juga dapat mengubah data Dokumen dengan klik tombol "Sunting" dan kembali ke daftar Dokumen dengan klik tombol "Daftar Dokumen". 

    \item \textbf{Mengubah Dokumen}\\
    Untuk mengubah data Dokumen, klik ikon pensil pada kolom tabel dan akan diarahkan ke halaman untuk melakukan ubah data. 

    \begin{figure}[H]
        \centering
        \includegraphics[width=12.45cm, height=5.75cm]{media/menu-dokumen/icon-pensil-Dokumen.png}
        \caption{Tampilan Icon Pensil Dokumen}
        \label{fig:bagan-alir}
    \end{figure}

    \begin{figure}[H]
        \centering
        \includegraphics[width=12.45cm, height=5.75cm]{media/menu-dokumen/edit-Dokumen.png}
        \caption{Tampilan Form Edit Dokumen}
        \label{fig:bagan-alir}
    \end{figure}

    Data yang dapat user ubah adalah kategori, nama, dokumen, dan Status Publikasi. Jika sudah, user klik tombol "Simpan" 
    
    \item \textbf{Hapus Dokumen}\\
    Untuk menghapus data Dokumen, klik ikon sampah pada kolom tabel dan akan muncul pesan untuk melakukan hapus data. 

    \begin{figure}[H]
        \centering
        \includegraphics[width=12.45cm, height=5.75cm]{media/menu-dokumen/hapus-Dokumen.png}
        \caption{Tampilan Konfirmasi Hapus Dokumen}
        \label{fig:bagan-alir}
    \end{figure}

\end{enumerate}